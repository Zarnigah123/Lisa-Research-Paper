\section{Milky way Model}
\label{sec:milky_way}
In this section, we will briefly outline the milky way galaxy model used in this study.
The model is developed by~\cite{wagg2021gravitational} and makes use of the galaxy's enrichment history by taking
into account the metallicity-radius-time relationship~\cite{Frankel2018}.
It uses a separate star formation history and spatial distribution for the \lowalpha, \highalpha\ discs, and bulge in the galaxy.

\subsection{Star formation rate}
\label{subsec:star_formation_rate}
The star formation rate for both the \lowalpha\ and \highalpha\ disks is expressed as,
\begin{equation}
    p(\tau) \propto \exp\left(-\frac{\tau_m - \tau}{\tau_\text{SFR}}\right),
    \label{eq:star_formation_rate_equation}
\end{equation}
where $\tau$ is the time difference between the star's ZAMS stage and today.
The age of milky way galaxy, $\tau_m$, is taken as \SI{12}{\giga\yr}, and the star formation rate as, $\tau_\text{
    SFR}\ $= \SI{6.8}{\giga\yr}.
The star-forming period of \lowalpha\ and \highalpha\ discs were taken as \SIrange{0}{8}{\giga\yr} and \SIrange{8}{12}{\giga\yr} respectively.
The model adopts \SIrange{6}{12}{\giga\yr} as the star-forming period of the bulge~\cite{Bovy2019}.

\subsection{Radial distribution}
\label{subsec:radial_distribution}
The radial distribution of stars within the milky way galaxy was performed using the following expression,
\begin{equation}%
    p(R) = \exp\left(-\frac{R}{R_d}\right)\frac{R}{R_d^2}
    \label{eq:radial_distribution_of_stars}
\end{equation}%

However, a different scale length, $R_d$, was chosen for each component of the galaxy.
For \lowalpha, the model uses $R_\text{exp}(\tau)$ as the scale length~\cite[Eq 6]{Frankel2018}, where

\begin{equation}%
    R_\text{exp}(\tau) = 4\,\text{kpc}\left[1 - \alpha_{R_\text{exp}}\left(\frac{\tau}{8\,\text{Gyr}}\right)\right],
    \label{eq:exponential_radius_equation}
\end{equation}%
with the value of inside-out growth parameter, $\alpha_{R_\text{exp}}$, as 0.3. For \highalpha\ disc and bulge, the value of scale length was chosen as $(1/0.43)\,$\si{\kpc} and \SI{1.5}{\kpc} respectively.

\subsection{Vertical distribution}
\label{subsec:vertical_distribution}
The model employs a similar method of single exponent expression with varying scale height parameters for the vertical distribution as well.
The exponential expression used is,
\begin{equation}
    p(|z|) = \frac{1}{z_d}\exp\left(-\frac{z}{z_d}\right),
    \label{eq:vertical_distribution_of_stars}
\end{equation}
where $z$ here is the vertical displacement from the galactic plane.
The scale height parameter, $z_d$, for \lowalpha, \highalpha\ and bulge was taken as \SI{0.3}{\kpc}~\cite{McMillan2011}, \SI{0.95}{\kpc}~\cite{Bovy2016}, and \SI{0.2}{\kpc}~\cite{Wegg2015} respectively.

\subsection{Metallicity-radius-time relationship}
\label{subsec:metallicity_radius_relationship}
The MRT relationship plays an important part, both in the galaxy model and later on in the placement of DCOs within the galaxy as well.
The model makes use of~\cite[Eq. 7]{Frankel2018},
\begin{equation}
\begin{aligned}
    \feh(R,\tau) &= F_m + \nabla\feh R \\
    &- \left(F_m + \nabla\feh R_{\feh=0}^{\text{now}}\right) f(\tau)\\
\end{aligned}
\end{equation}
For each point generated, if the value of metallicity produced by the MW model was less or greater than the limits defined by COMPAS\footnote{0.0001, 0.03} it was changed to a uniformly drawn random number between $\text{COMPAS}_\text{min} - \text{ZSOLAR}$ and $\text{ZSOLAR} - \text{COMPAS}_\text{max}$ respectively.
\subsection{Galaxy synthesis}
\label{subsec:galaxy_synthesis}
For the synthesis of an instance of the Milky Way galaxy, the model described previously samples the following parameters,
\begin{equation*}
    \theta_i = \{\tau, D, z, \text{component}\},
\end{equation*}
where $\tau$ is the look-back time for the binary, $D$ is the distance from Earth, $z$ is the metallicity, and `component' is the component of the galaxy in which the binary resides.\footnote{One of the three, \lowalpha\ disc, \highalpha\ disc, or bulge.}
The parameters are generated for $i = 1, 2, 3, \ldots$, $N_\text{GAL}$, where $N_\text{GAL} = 100$.